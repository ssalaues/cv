%%%%%%%%%%%%%%%%%%%%%%%%%%%%%%%%%%%%%%%%%
% Developer CV
% LaTeX Template
% Version 1.0 (28/1/19)
%
% This template originates from:
% http://www.LaTeXTemplates.com
%
% Authors:
% Jan Vorisek (jan@vorisek.me)
% Based on a template by Jan Küster (info@jankuester.com)
% Modified for LaTeX Templates by Vel (vel@LaTeXTemplates.com)
%
% License:
% The MIT License (see included LICENSE file)
%
%%%%%%%%%%%%%%%%%%%%%%%%%%%%%%%%%%%%%%%%%

%----------------------------------------------------------------------------------------
%	PACKAGES AND OTHER DOCUMENT CONFIGURATIONS
%----------------------------------------------------------------------------------------

\documentclass[10pt]{developercv} % Default font size, values from 8-12pt are recommended

%----------------------------------------------------------------------------------------

\begin{document}

%----------------------------------------------------------------------------------------
%	TITLE AND CONTACT INFORMATION
%----------------------------------------------------------------------------------------

\begin{minipage}[t]{0.45\textwidth} % 45% of the page width for name
	\vspace{-\baselineskip} % Required for vertically aligning minipages
	
	% If your name is very short, use just one of the lines below
	% If your name is very long, reduce the font size or make the minipage wider and reduce the others proportionately
	\colorbox{black}{{\HUGE\textcolor{white}{\textbf{\MakeUppercase{Salim}}}}} % First name
	
	\colorbox{black}{{\HUGE\textcolor{white}{\textbf{\MakeUppercase{Salaues}}}}} % Last name
	
	\vspace{6pt}
	
	{\huge Dev Ops Engineer} % Career or current job title
\end{minipage}
\begin{minipage}[t]{0.3\textwidth} % 27.5% of the page width for the first row of icons
	\vspace{-\baselineskip} % Required for vertically aligning minipages
	
	% The first parameter is the FontAwesome icon name, the second is the box size and the third is the text
	% Other icons can be found by referring to fontawesome.pdf (supplied with the template) and using the word after \fa in the command for the icon you want
	\icon{MapMarker}{12}{San Francisco, CA}\\
	\icon{At}{12}{\href{mailto:salimsalaues@gmail.com}{salimsalaues@gmail.com}}\\

\end{minipage}
\begin{minipage}[t]{0.275\textwidth} % 27.5% of the page width for the second row of icons
	\vspace{-\baselineskip} % Required for vertically aligning minipages
	
	% The first parameter is the FontAwesome icon name, the second is the box size and the third is the text
	% Other icons can be found by referring to fontawesome.pdf (supplied with the template) and using the word after \fa in the command for the icon you want
	% \icon{Globe}{12}{\href{https://alyx.vance.me}{alyx.vance.me}}\\
	\icon{Github}{12}{\href{https://github.com/ssalaues}{github.com/ssalaues}}\\
	\icon{Phone}{12}{+1 661 713 8213}\\
	% \icon{Twitter}{12}{\href{https://twitter.com/@alyxvance}{@alyxvance}}\\
\end{minipage}

\vspace{0.5cm}

%----------------------------------------------------------------------------------------
%	INTRODUCTION, SKILLS AND TECHNOLOGIES
%----------------------------------------------------------------------------------------

\begin{minipage}[t]{0.45\textwidth} % 40% of the page width for the introduction text
	\vspace{-\baselineskip} % Required for vertically aligning minipages
	\cvsect{Who Am I?}

	I'm a self-motivated, creative, and curious individual who likes to keep up with the latest
	tech. With my love for knowledge sharing, I keep in mind that as an engineer I don't live in a vaccuum
	so it's important to try to make complex systems easily accessible and understandable to others. \\
\end{minipage}
\hfill  % Whitespace between
\begin{minipage}[t]{0.45\textwidth} % 50% of the page for the skills bar chart
	\vspace{-\baselineskip} % Required for vertically aligning minipages
	\cvsect{Proficiencies}
	\begin{barchart}{5.5}
		\baritem{Python}{70}
		\baritem{Bash}{90}
		\baritem{Terraform}{80}
		\baritem{YAML}{90}
		\baritem{Git}{80}
	\end{barchart}
\end{minipage}

% \begin{center}
% 	\bubbles{6/Linux, 5/git, 4/Office, 3/Inkscape, 3/Blender}
% \end{center}

%----------------------------------------------------------------------------------------
%	EXPERIENCE
%----------------------------------------------------------------------------------------

\cvsect{Experience}

\begin{entrylist}
	\entry
		{2019 -- Present}
		{\Large Lead Dev Ops Engineer}
		{\large OpenInvest}
		{\small Initially solving startup scaling problems through migrations to more robust infrastructure in \mbox{Kubernetes} and industry best practices. Built out
		internal tooling to not only optimize developer workflows but also work with operations to automate complex proceedures. Implented full stack monitoring, 
		tracing, and alerting alongside dashboards to give developers more clear visibily across the stack for ease of debugging. Eventually lead post-acquisition
		infrastructure migrations to J.P. Morgan Chase internal tooling. Current ongoing duties include managing evolving infrastructure requirements, maintaining
		security compliance and either delgation of work or pairing to knowledge share with more junior DevOps engineers. Constantly context switching to deliver
		daily unique challenges that may arise. \\
		\texttt{Kubernetes}\slashsep\texttt{Python}\slashsep\texttt{Terraform}\slashsep\texttt{AWS}\slashsep\texttt{NodeJS}}
	\entry
		{2017 -- 2019}
		{\Large Dev Ops Engineer}
		{\large Scality}
		{\small Lead maintainer for multi-cloud object storage (decoupled microservice-based) infrastructure
		deployed via templated Kubernetes manifests (Helm) with Ansible for Kubernetes deployment (Kubespray).
		Setup CI/CD pipeline and infrastructure to test all microservices in an end to end environment. Regularly
		researching and implementing POC solutions for incoming problems. Responsible for CI tooling in Python,
		feature development in GoLang, and legacy code in NodeJS.\\
		\texttt{Kubernetes}\slashsep\texttt{GoLang}\slashsep\texttt{Ansible}\slashsep\texttt{AWS}\slashsep\texttt{Python}\slashsep\texttt{NodeJS}}
	\entry
		{2016 -- 2017}
		{\Large Site Reliability Engineer}
		{\large Ecole 42}
		{\small Management of thousands of iMac desktops alongside dozens of Linux-based servers in a mixed ecosystem using Ansible for state. Worked around
		the clock with a team to ensure that all user facing and underlying systems were fully operational to meet the demands of a rapidly growing school. \\
		\texttt{Python}\slashsep\texttt{Ansible}\slashsep\texttt{Linux}}
	\entry
		{2016 -- 2017}
		{\Large Systems Engineer}
		{\large Aero Kinetics}
		{\small Infrastructure design and deployment for a commercial drone startup integrating the server side services with client side web apps.\\\texttt{Java}\slashsep\texttt{Windows Server}}
	\entry
		{2013 -- 2016}
		{\Large Systems Administrator}
		{\large F1 Information Technologies}
		{\small Remote admin managing ADs, on-premise/cloud Exchange, and supporting local/remote users across domestic and international clients.}


\end{entrylist}

%----------------------------------------------------------------------------------------
%	EDUCATION
%----------------------------------------------------------------------------------------

% \cvsect{Education}

% \begin{entrylist}
% 	\entry
% 		{2016 -- 2017}
% 		{Ecole 42}
% 		{A French engineering school specializing in Computer Science.}

% \end{entrylist}

%----------------------------------------------------------------------------------------
%	ADDITIONAL INFORMATION
%----------------------------------------------------------------------------------------

\begin{minipage}[t]{0.25\textwidth}
	\vspace{-\baselineskip} % Required for vertically aligning minipages
	\cvsect{Languages}
	
	\textbf{English} - native\\
	\textbf{Spanish} - proficient\\
	\textbf{French} - intermediate
\end{minipage}
\hfill
\begin{minipage}[t]{0.25\textwidth}
	\vspace{-\baselineskip} % Required for vertically aligning minipages
	\cvsect{Hobbies}
	
	\footnotesize A passion for photography, comic books conventions, and the \mbox{occasional} cars \& coffee.
\end{minipage}
\hfill
\begin{minipage}[t]{0.25\textwidth}
	\vspace{-\baselineskip} % Required for vertically aligning minipages
	\cvsect{Non profit}

	\footnotesize Mentor for kids teaching Arduino workshops and helping them understand
	basics in a fun and engaging manner.
\end{minipage}

%----------------------------------------------------------------------------------------

\end{document}
